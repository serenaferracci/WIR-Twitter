\documentclass[journal,11pt]{vgtc}  

\usepackage{mathptmx}
\usepackage{graphicx}
\usepackage{times}

\usepackage[bookmarks,backref=true,linkcolor=black]{hyperref} %,colorlinks
\hypersetup{
  pdfauthor = {},
  pdftitle = {},
  pdfsubject = {},
  pdfkeywords = {},
  colorlinks=true,
  linkcolor= black,
  citecolor= black,
  pageanchor=true,
  urlcolor = black,
  plainpages = false,
  linktocpage
}

\onlineid{0}


\title{Twitter Trending Topic Classification}
\author{MES: Maria Ludovica Costagliola - Emanuele De Santis - Serena Ferracci}

%% Abstract section.
\abstract{
  \normalsize
  The Web Information Retrieval course allowed us to develope the described project.
  It is about classification of topics that are trending of twitter in a given moment. We have dealt 
  with (i) text based classification using single tweets (ii) network based classification exploiting the 
  graph structure of Twitter. 

  The used dataset is composed by tweets divided by trending topics. 
  We classify the trending topics into 8 categories: \textit{Event, Health, Movie, Music, Politics, Science, Society} and \textit{Sport}.

  The first approach gives good result in terms of precision and recall. Instead, the second approach 
  does not give conclusive results because it require too many resources to be implemented on a single machine.

} 


\begin{document}


\maketitle

\section{Introduction}
We chose this project because Twitter is one of the most popular social network. 
It is used every day by millions of users expressing their opinions about several fields.
When an user looks for something, the first thing Twitter displays to him is the list of 
trending topics of the moment. Often the user can not know what the topic is about, so it as to manually 
search for tweets belonging to that trending topic to better understand it.

It is interesting for the user to have a way to know what the Trending Topic is about
without further searches.
Our work tries to replicate the results obtained by \cite{lee_palsetia_narayanan_patwary_agrawal_choudhary_2011}.  
The general categories used in this project are 8, named: \textit{Event, Health, Movie, Music, Politics, Science, Society} and \textit{Sport}.


\section{Related Work}
The paper we refer to is the work done by Lee et al. for tweets classification \cite{lee_palsetia_narayanan_patwary_agrawal_choudhary_2011}. 
They have used 18 general categories to classify each trending topic.
They address the problem following two different approaches 



\bibliographystyle{abbrv}
%%use following if all content of bibtex file should be shown
%\nocite{*}
\bibliography{template}
\end{document}
